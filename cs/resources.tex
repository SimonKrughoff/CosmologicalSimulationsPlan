% ====================================================================
\section{Resource requirements}
\resetnumbering
\label{sec:resources}
% ====================================================================


% --------------------------------------------------------------------

\subsection{Introduction}
\label{sec:resources:introduction}
In this section we elaborate on the resources needed to deliver the simulations and catalogs described in the previous section. From a development perspective, the main tasks lay within generating and validating the catalogs from the simulations. The distribution of the resulting catalogs can also amount to a non-trivial task that could require considerable resources. The DESC collaboration has to evaluate how important this will be and decide what resources to allocate to this task. Finally, some of the tasks are cross-cutting different surveys. For example the generation of simulations for covariances is important not only for DESC, but also for DESI, WFIRST, and Euclid. First discussions between these surveys have been initiated and sharing resources between them seems a logical step.

\subsection{HPC Resources}

HPC resources used within the collaboration are of several types. The DOE Leadership Computing Facilities offer two venues for applying for very large computing allocations: the INCITE grants and the ALCC grants. The INCITE grants are open to all institutions, national, international, DOE, NSF, NASA and beyond. The two main criteria for an award-winning proposal are outstanding science problems and highly scalable codes (one has to demonstrate scalability to at least 20\% of  machine). The Argonne group has been successful in applying for INCITE time in the past and these allocations enabled them to generate the large simulations underlying DC1 and DC2. The ALCC allocations include also the NERSC supercomputers. Here, the allocation time is controlled by the DOE Office of Science program managers and therefore a strong connection to DOE science problems is required for these project. Scalable code is also essential for successful proposal. Argonne, LBNL, and Fermilab have been successful in applying for resources under the ALCC allocations.

In addition, due to the advent of three new leadership computers, Cori at NERSC, Theta at the ALCF (followed later by Aurora) and Summit at the OLCF, separate calls for Early Science Projects on these machines were announced. Both LBNL and Argonne were awarded time on Cori and Argonne also secured time on Theta and Summit. The Argonne simulations will target the hydrodynamics tasks outlined as part of DC3. Given the past success, we are confident that we will be able to also obtain time on these supercomputers to carry out the large gravity-only simulations needed for DC3.

Besides these targeted proposals, DOE also awards computing time specifically directed to the projects on NERSC resources. LSST DESC has been able to able to obtain time on the NERSC machines under this application. Early on, some of this time was used to carry out the DM simulation that is currently used for the mock comparison project. In the future, some of this time might be used for specific analysis and mock creation tasks if it is available.

Some groups within DESC (mainly the CMU group) have also successfully applied for XSEDE time awarded by NSF. This time has been mostly used for hydrodynamics simulations. 

To summarize, for DC1 there are no additional computing needs. For DC2 the resources needed are mainly for analysis tasks and mock catalog generation tasks, the major simulations have been either already carried out (gravity-only) or resources have been secured (hydrodynamics). The computing time for analysis is relatively small and should be available as either part of ALCC allocations or the DESC NERSC allocations. For DC3, we have secured computing allocations for the hydrodynamics simulations and we are confident that we will be able to write a winning proposal for the gravity-only simulations. Again, computing time for analysis tasks and mock generation should be covered by the NERSC allocation for DESC and possibly an ALCC allocation.

One major piece that is currently missing is an allocation for covariance studies. It is not clear however if this will be needed before the 2018/19 time frame by when the DOE computing facilities have grown by more than an order of magnitude and should be able to provide the needed time.

\subsection{Personnel Effort}

{\bf Simulations:} 

{\bf Mock catalog generation and validation:}

{\bf Distribution, Handling, and Manipulation:}

\subsection{Responsibility for Major Simulation Deliverables}

The major groups involved in the efforts described in this document are (in alphabetical order): Argonne, Brazil, CMU, LBNL, Stanford, and UW. {\bf who else? how detailed?}  

% ====================================================================

\vspace{\baselineskip}
\hrule
\clearpage
